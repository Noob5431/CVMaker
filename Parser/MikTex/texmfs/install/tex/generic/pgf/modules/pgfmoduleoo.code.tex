% Copyright 2019 by Till Tantau
% Copyright 2019 by Saso Zivanovic
%
% This file may be distributed and/or modified
%
% 1. under the LaTeX Project Public License and/or
% 2. under the GNU Public License.
%
% See the file doc/generic/pgf/licenses/LICENSE for more details.
%
% Based on pgfmoduleoo.code.tex v1.8 by Till Tantau.  This version adds inheritance.



% Support of object-oriented programming in TeX, with inheritance.


% The oo support works as follows:
%
% The main supported concepts are classes, objects, methods,
% attributes and signal/slots. A class defines a set of methods, which are,
% in the end, just normal TeX macros. Once a class has been created,
% it can be instantiated by calling the \pgfoonew command, resulting
% in a new object. Objects are local to their group. Given an object,
% you can send it a message, resulting in the method code of the
% object's method to be executed. While an object exists, it has a set
% of attributes whose values can change over time. Attributes values
% are not local to TeX groups, rather their life-cycle is the same as
% the object's (which, however, is local to the group in which the
% object was declared).
%
% The implementation is as follows: There is an ID counter that is
% increased each time an object is created. This counter is local to
% the group, which means that when a group ends the counter will
% revert to the previous value, allowing objects to the reused in
% subsequent groups.
%
% A method is just a normal \TeX macro, but to call it you write
% \objecthandle.methodname(parameters). The \objecthandle is a macro
% that is created when you say \pgfoonew. The special object
% \pgfoothis is the current object.
%
% Attributes are stored globally in internal TeX macros whose name is
% composed of the object number and the attribute name.




% Declares a class
%
% #1 = class name
% #2 = methods
%
% Description:
%
% This command declares a class for future use. Inside #2, the macro
% \method can be used to declare a method. The \method macro takes a
% method name, parameters (methods are normal TeX macros, after all)
% and body.

% base classes are put in parenthesis before the new class name, but
% are optional.
\def\pgfooclass{%
  \pgfutil@ifnextchar ({%
    \pgfooclass@
  }{%
    \pgfooclass@()
  }%
}%
\long\def\pgfooclass@(#1)#2#3{%
  \def\pgfoo@classname{#2}%
  \expandafter\ifx\csname pgfooY\pgfoo@classname.@pgfooinit\endcsname\relax\else
    \pgferror{class #2 is already defined}%
  \fi
  \pgfoo@ciii{#2}{#1}%
  \expandafter\let\csname pgfooY#2@pgfoo@mro\endcsname\pgfoo@mro
  \let\pgfoo@origmethod=\method%
  \let\pgfoo@origattribute=\attribute%
  \let\method=\pgfoo@declaremethod%
  \let\attribute=\pgfoo@declareattribute%
  \let\pgfoo@attributes=\pgfutil@empty%
  \let\pgfoo@methods=\pgfutil@empty%
  #3%
  % inherit
  \def\pgfoo@temp@baseclasses{#1}%
  \pgfoo@inherit@methods
  \pgfoo@inherit@attributes
  % Always present (and never inherited) methods:
  \expandafter\let\csname pgfooY\pgfoo@classname.get handle\endcsname\pgfoo@obj%
  \expandafter\let\csname pgfooY\pgfoo@classname.get id\endcsname\pgfoo@id%
  % for compatibility with previous versions of pgfoo, define method
  % with the same name as the class as a synonym for init
  \expandafter\let\expandafter\pgfoo@init\csname pgfooY\pgfoo@classname.init\endcsname
  \expandafter\let\csname pgfooY\pgfoo@classname.\pgfoo@classname\endcsname\pgfoo@init
  % Cleanup
  \let\method=\pgfoo@origmethod%
  \let\attribute=\pgfoo@origattribute%
}%

\def\pgfoo@emptyinit(){}%
\def\pgfoo@escapeif#1#2\fi{\fi#1}% a little helper


%%%%%%%%%%%%%%%%%%%%%%%%%%%%%%%%%%%%%%%%%%%%%%%%%%%%%%%%%%%%%%%%%%%%%%
% Compute MRO using C3 algorithm
% The sequences are enclosed in brackets; elements of sequences and
% the big list are terminated by a comma and a dot, respectively. (Why
% comma *and* a dot: TeX's parset---brackets don't form groups! Soooo,
% why not braces? Because I don't know how do delete excess "{},"s...)

% #1 = current class, #2 = list of base classes
\def\pgfoo@ciii#1#2{%
  % \pgfoo@seq will hold a list of sequences.
  % (i) the first sequence has just one element: the current class
  \def\pgfoo@seq{[#1,].}%
  % (ii) every subclass has a sequence (in the order as they were
  % given): the class' sequence is its mro.
  \pgfutil@for\pgfoo@baseclass:={#2}\do{%
    \edef\pgfoo@baseclass@mro{\csname pgfooY\pgfoo@baseclass @pgfoo@mro\endcsname}%
    \expandafter\ifx\expandafter\relax\pgfoo@baseclass@mro\relax\else
      \edef\pgfoo@seq{\pgfoo@seq[\pgfoo@baseclass@mro,].}%
    \fi
  }%
  % The final sequence is a sequence of base classes, in the order as
  % they were given.
  \ifx\relax#2\relax\else
    \edef\pgfoo@seq{\pgfoo@seq[#2,].}%
  \fi
  \edef\pgfoo@temp{()(\pgfoo@seq)}%
  \expandafter\pgfoo@ciii@merge\pgfoo@temp
}%

% #1 = MRO so far
% #2 = remaining sequences
% no empty sequences [] should arrive here: they should be removed
% before calling this macro
\def\pgfoo@ciii@merge(#1)(#2){%
  \ifx\relax#2\relax
    \pgfoo@ciii@done(#1)%
  \else\pgfoo@escapeif{% split #2
    \pgfoo@ciii@merge@A(#1)(#2)%
  }\fi%
}%
% a splitter: to be only called from \pgfoo@ciii@merge
\def\pgfoo@ciii@merge@A(#1)(#2.#3){%
  \ifx\relax#3\relax % if #3 is empty, we can take the shortcut
    \pgfoo@ciii@mergeA@done(#1)(#2)%
  \else\pgfoo@escapeif{%
    \pgfoo@ciii@merge@checkHead(#1)()(#2)()(#3)%
  }\fi
}%

\def\pgfoo@ciii@mergeA@done(#1)([#2]){%
  \pgfoo@ciii@done(#1#2)%
}%

% #1 = MRO so far
% #2 = sequences with bad heads (dot after each sequence)
% [#3,#4] = the sequence whose head we're testing right
%           now---obviously, it should not be empty! (#3 has no comma,
%           #4 has a comma after each class name)
% #5 = #3 does not occur in the tail of these sequences (dot after
%      each sequence)
% #6 = the remaining sequences (dot after each sequence)
\def\pgfoo@ciii@merge@checkHead(#1)(#2)([#3,#4])(#5)(#6){%
  \pgfutil@in@{[#3,}{#2}%
  \ifpgfutil@in@\pgfoo@escapeif{% #3 is already blacklisted
    \pgfoo@ciii@clean\pgfoo@ciii@merge@checkHead@P{(#1)(#2[#3,#4].)(#5#6)}%
  }\else\pgfoo@escapeif{%
    \ifx\relax#6\relax\pgfoo@escapeif{% no more sequences to test the head against!
                                 % so, #3 is a good head, and we can recurse
      \pgfoo@ciii@goodHeadFound(#1)(#3)(#2[#4].#5)%
    }\else\pgfoo@escapeif{% split #6 and get to work
      \pgfoo@ciii@merge@checkHead@A(#1)(#2)([#3,#4])(#5)(#6)%
    }\fi
  }\fi
}%
% splitter: to be only called from \pgfoo@ciii@merge@checkHead
% we will check if #3 occurs in the tail of #6
\def\pgfoo@ciii@merge@checkHead@A(#1)(#2)([#3,#4])(#5)(#6.#7){%
  \pgfoo@ciii@merge@checkHead@B(#1)(#2)([#3,#4])(#5)(#6)(#7)%
}%
% splitter: to be only called from \pgfoo@ciii@merge@checkHead@A
% &worker: we will check if #3 occurs in #7. If it does, #3 is a bad head.
\def\pgfoo@ciii@merge@checkHead@B(#1)(#2)([#3,#4])(#5)([#6,#7])(#8){%
%  \def\pgfoo@ciii@tempA{#3}\def\pgfoo@ciii@tempB{#5}%
%  \ifx\pgfoo@ciii@tempA\pgfoo@ciii@tempB
%    \pgferror{An attempt to derive from the same class twice}%
%  \else\pgfoo@escapeif{%
    \pgfutil@in@{,#3,}{,#7}%
    \ifpgfutil@in@\pgfoo@escapeif{% bad head! get next head and restart
                             % but: need to clean up first!
      \pgfoo@ciii@clean\pgfoo@ciii@merge@checkHead@P{(#1)(#2[#3,#4].)(#5[#6,#7].#8)}%
    }\else\pgfoo@escapeif{% so far, so good
      \ifx\relax#8\relax\pgfoo@escapeif{% if #8 is empty, we won:
        % #3 is a good head! we will move it to MRO
        % but clean-up first!
        \pgfoo@ciii@goodHeadFound(#1)(#3)(#2[#4].#5[#6,#7].)%
      }\else\pgfoo@escapeif{% proceed to the first seq in #8
        \pgfoo@ciii@clean\pgfoo@ciii@merge@checkHead{(#1)(#2)([#3,#4])(#5[#6,#7].)(#8)}%
      }\fi
    }\fi
%  }\fi
}%

% #1 = MRO
% #2 = good head
% #3 = the remaining sequences (in need of removing the good head from
% them, and then also some general cleanup)
\def\pgfoo@ciii@goodHeadFound(#1)(#2)(#3){%
  \pgfutil@in@{,#2,}{#3}%
  \ifpgfutil@in@\pgfoo@escapeif{%
    \def\pgfoo@ciii@removeGoodHead(##1)(##2)(##3,#2,##4){%
      \pgfoo@ciii@goodHeadFound(##1)(##2)(##3,##4)}%
    \pgfoo@ciii@removeGoodHead(#1)(#2)(#3)%
  }\else\pgfoo@escapeif{%
    \pgfutil@in@{[#2,}{#3}%
    \ifpgfutil@in@\pgfoo@escapeif{%
      \def\pgfoo@ciii@removeGoodHead(##1)(##2)(##3[#2,##4){%
        \pgfoo@ciii@goodHeadFound(##1)(##2)(##3[##4)}%
      \pgfoo@ciii@removeGoodHead(#1)(#2)(#3)%
    }\else\pgfoo@escapeif{%
      \pgfoo@ciii@clean\pgfoo@ciii@merge{(#1#2,)(#3)}%
    }\fi
  }\fi
}%

% mediates between the bad head exit of \pgfoo@ciii@merge@checkHead@B
% and \pgfoo@ciii@merge@checkHead; we need the mediator because #3
% below might have been affected by cleaning.
\def\pgfoo@ciii@merge@checkHead@P(#1)(#2)(#3){%
  \if\relax#3\relax % no more good head candidates!
    \pgferror{Bad MRO: There is no good (monotonic etc.)
      Method Resolution Order for your class hierarchy}%
  \else\pgfoo@escapeif{%
    \pgfoo@ciii@merge@checkHead@Q(#1)(#2)(#3)%
  }\fi
}%
% splitter for \pgfoo@ciii@merge@checkHead@P
% #3 = the sequence containing the newest good head candidate
\def\pgfoo@ciii@merge@checkHead@Q(#1)(#2)(#3.#4){%
  \pgfoo@ciii@merge@checkHead(#1)(#2)(#3)()(#4)%
}%

% cyclically removes all occurrences of "[]," and "[]." (i.e., empty
% sequences) in #2. When done, calls #1 with whatever remains.
\def\pgfoo@ciii@clean#1#2{% clean commas
  \pgfutil@in@{[].}{#2}% clean dots
  \ifpgfutil@in@\pgfoo@escapeif{%
    \pgfoo@ciii@clean@dot#1#2\pgfoo@ciii@clean@END
  }\else\pgfoo@escapeif{%
    #1#2%
  }\fi
}%
\def\pgfoo@ciii@clean@dot#1#2[].#3\pgfoo@ciii@clean@END{%
  \pgfoo@ciii@clean#1{#2#3}%
}%

\def\pgfoo@ciii@done(#1,){%
  \def\pgfoo@mro{#1}%
}%

% end of C3
%%%%%%%%%%%%%%%%%%%%%%%%%%%%%%%%%%%%%%%%%%%%%%%%%%%%%%%%%%%%%%%%%%%%%%


% Declare a method
%
% Description:
%
% Defines a method. To use/invoke this method for an object \object,
% you write \object.methodname(parameters). This will cause the method
% body to be invoked with the argument "(parameters)".
%
% To define the method \method should be directly followed by the
% method name and, then, by (, followed by a parameter pattern,
% followed by ), followed by the message body. Spaces are allowed only
% after "\method" and after the closing ).
%
% Example:
%
% \pgfooclass{MyPlot}{
%
%   \attribute x=0;
%   \attribute y=0;
%
%   \method MyPlot() {
%   }
%
%   \method getX(#1) {
%     \pgfooget{x}{#1}
%   }
%
%   \method setPoint(#1,#2) {
%     \pgfooset{x}{#1}
%     \pgfooset{y}{#2}
%   }
% }

%   define the method macro and append the method to the method collection
\long\def\pgfoo@declaremethod#1(#2)#3{%
  \def\pgfoo@method{#1}%
  \ifx\pgfoo@classname\pgfoo@method
    \def\pgfoo@method{init}%
  \fi
  \expandafter\long\expandafter\def\csname pgfooY\pgfoo@classname.\pgfoo@method\endcsname(#2){#3}%
  \edef\pgfoo@methods{\pgfoo@methods,\pgfoo@method}%
}%

\def\pgfoo@inherit@methods{%
  % for non-derived classes:
  \expandafter\ifx\expandafter\relax\pgfoo@temp@baseclasses\relax
    % if init is not defined, define an empty one
    \expandafter\ifx\csname pgfooY\pgfoo@classname.init\endcsname\relax
      \expandafter\let\csname pgfooY\pgfoo@classname.init\endcsname\pgfoo@emptyinit%
      \edef\pgfoo@methods{,init\pgfoo@methods}%
    \fi
  \fi
  \if\relax\pgfoo@methods\relax\else
    % gobble the initial comma
    \edef\pgfoo@methods{\expandafter\pgfutil@gobble\pgfoo@methods}%
  \fi
  % remember the list of collected methods
  \expandafter\let\csname pgfooY\pgfoo@classname @pgfoo@methods\endcsname\pgfoo@methods
  % get MRO
  \expandafter\let\expandafter\pgfoo@mro\csname pgfooY\pgfoo@classname @pgfoo@mro\endcsname
  % loop throough base classes in MRO
  \pgfutil@for\pgfoo@baseclass:=\pgfoo@mro\do{%
    \ifx\pgfoo@baseclass\pgfoo@classname % skip self
    \else\pgfoo@escapeif{%
      % get methods defined in this base class
      \expandafter\let\expandafter\pgfoo@methods\csname pgfooY\pgfoo@baseclass @pgfoo@methods\endcsname
      \pgfutil@for\pgfoo@method:=\pgfoo@methods\do{% for each method
        % check if it is already defined in our class
        \expandafter\ifx\csname pgfooY\pgfoo@classname.\pgfoo@method\endcsname\relax
          % if not, link the method name from our class to base class
          \edef\pgfoo@temp{\noexpand\let
            \expandafter\noexpand\csname pgfooY\pgfoo@classname.\pgfoo@method\endcsname
            \expandafter\noexpand\csname pgfooY\pgfoo@baseclass.\pgfoo@method\endcsname}%
          \pgfoo@temp
        \fi
      }%
    }\fi
  }%
}%

\def\pgfoosuper(#1,#2).#3({%
  #2.get id(\pgfoo@temp@id)%
  \edef\pgfoo@classname{\csname pgfooX\pgfoo@temp@id @class\endcsname}%
  \def\pgfoo@super@A##1,#1,##2.##3({%
    \pgfoo@super@B##2.##3(%
  }%
  \expandafter\let\expandafter\pgfoo@mro\csname pgfooY\pgfoo@classname @pgfoo@mro\endcsname
  \expandafter\pgfoo@super@A\expandafter,\pgfoo@mro,.#3(%
}%
\def\pgfoo@super@B#1,#2.#3({%
  \expandafter\ifx\csname pgfooY#1.#3\endcsname\relax
    \pgfoo@escapeif{\pgfoo@super@B#2.#3(}%
  \else
    \pgfoo@escapeif{\expandafter\pgfoo@caller\pgfoo@temp@id.#1.#3(}%
  \fi
}%



% Declare an attribute
%
% #1 = attribute name
% #2 = optional default value
%
% Description:
%
% Declares the attribute #1 for the current class. If the attribute
% name is followed by =, the text following the equal sign is the
% default value.

\def\pgfoo@declareattribute#1;{%
  \pgfutil@in@{ =}{#1}%
  \ifpgfutil@in@%
    \pgfoo@declareunpackspace#1;%
  \else%
    \pgfutil@in@={#1}%
    \ifpgfutil@in@%
      \pgfoo@declareunpack#1;%
    \else%
      \pgfoo@declareattribute@\let{#1}\pgfutil@empty
    \fi%
  \fi%
}%
\def\pgfoo@declareunpack#1=#2;{\pgfoo@declareattribute@\def{#1}{{#2}}}%
\def\pgfoo@declareunpackspace#1 =#2;{\pgfoo@declareattribute@\def{#1}{{#2}}}%

% put the initial value in the class's namespace
% put the attr name in a list
\def\pgfoo@declareattribute@#1#2#3{%
  \expandafter#1\csname pgfooY\pgfoo@classname @#2\endcsname#3%
  \expandafter\def\expandafter\pgfoo@attributes\expandafter{\pgfoo@attributes,#2}%
}%

% for each base class:
%   for each attribute declared in that class:
%       add the attr on a list if it's not there yet
% convert the list of (attr,class which it comes from) into an
% efficient sequence of triples \op{attrname}\pointer-to-initial-value
% (\op is then set differently in init and garbage collection)
\def\pgfoo@inherit@attributes{%
  % store the list of attributes declared in this class for use in
  % derived classes
  \if\relax\pgfoo@attributes\relax\else
    \edef\pgfoo@attributes{\expandafter\pgfutil@gobble\pgfoo@attributes}% gobble comma
  \fi
  \expandafter\let\csname pgfooY\pgfoo@classname.@pgfoo@attributes\endcsname\pgfoo@attributes%
  % get MRO
  \expandafter\let\expandafter\pgfoo@mro\csname pgfooY\pgfoo@classname @pgfoo@mro\endcsname
  \def\pgfoo@allattributes{}% here we will store the attrs declared in base classes
  \pgfutil@for\pgfoo@baseclass:=\pgfoo@mro\do{% for each base class in MRO order
    % the attributes declared in this base class
    \expandafter\let\expandafter\pgfoo@attributes\csname pgfooY\pgfoo@baseclass.@pgfoo@attributes\endcsname
    \pgfutil@for\pgfoo@attribute:=\pgfoo@attributes\do{% for each attribute
      % check if we have already found it in some previous base class
      \edef\pgfoo@temp{{,\pgfoo@attribute=}{\pgfoo@allattributes,}}%
      \expandafter\pgfutil@in@\pgfoo@temp
      \ifpgfutil@in@\else % don't overwrite
        % append to the list
        \edef\pgfoo@allattributes{\pgfoo@allattributes,\pgfoo@attribute=\pgfoo@baseclass}%
      \fi
    }%
  }%
  \if\relax\pgfoo@allattributes\relax\else
    \edef\pgfoo@allattributes{\expandafter\pgfutil@gobble\pgfoo@allattributes}% gobble comma
  \fi
  % this macro will hold a list of attributes from all base classes
  % we run this macro at init and gc
  \def\pgfoo@process@attributes{}%
  \pgfutil@for\pgfoo@attributeclass:=\pgfoo@allattributes\do{% for each (attribute, class) pair
    % append to the attribute processor
    \expandafter\pgfoo@inherit@attributes@appendattribute\pgfoo@attributeclass.%
  }%
  \expandafter\let\csname pgfooY\pgfoo@classname .@pgfoo@process@attributes\endcsname\pgfoo@process@attributes
}%


% #1=attr, #2=class which the attr comes from
\def\pgfoo@inherit@attributes@appendattribute#1=#2.{%
  \def\pgfoo@tempA{\pgfoo@attribute@op{#1}}%
  \expandafter\expandafter\expandafter\def\expandafter\expandafter\expandafter\pgfoo@tempB\expandafter\expandafter\expandafter{\expandafter\pgfoo@tempA\csname pgfooY#2@#1\endcsname}%
  \expandafter\expandafter\expandafter\def\expandafter\expandafter\expandafter\pgfoo@process@attributes\expandafter\expandafter\expandafter{\expandafter\pgfoo@process@attributes\pgfoo@tempB}%
  % above is just this:
  %     \def\pgfoo@process@attributes{\pgfoo@process@attributes\pgfoo@attribute@op{#1}%
  %     \csname pgfooY#2@#1\endcsname}
  %  with \pgfoo@process@attributes and \csname expanded exactly once:
}%

% \pgfoo@attribute@op for garbage collection
\def\pgfoo@gc@attribute#1#2{\pgfoolet{#1}\relax}%


\newcount\pgfoo@objectcount
\newcount\pgfoothis@count

% The attribute value method
%
% #1 = attr
%
% This method inserts the current value of the given attribute for the
% current object.

\def\pgfoovalueof#1{%
  \csname pgfooX\the\pgfoothis@count @#1\endcsname%
}%


% The attribute get method
%
% #1 = attr
% #2 = macro
%
% This method makes the macro equal to the current value of the
% attribute for the current object.

\def\pgfooget#1#2{%
  \expandafter\let\expandafter#2\csname pgfooX\the\pgfoothis@count @#1\endcsname%
}%


% The attribute set method
%
% #1 = attr
% #2 = value
%
% This method sets the given attribute for the current object to the
% given value.

\long\def\pgfooset#1#2{%
  \expandafter\gdef\csname pgfooX\the\pgfoothis@count @#1\endcsname{#2}%
}%


% The attribute set method (expanded version)
%
% #1 = attr
% #2 = value
%
% This method sets the given attribute for the current object to the
% expanded version of the given value.

\long\def\pgfooeset#1#2{%
  \expandafter\xdef\csname pgfooX\the\pgfoothis@count @#1\endcsname{#2}%
}%


% The attribute let method
%
% #1 = attr
% #2 = value
%
% This method sets the given attribute for the current object to the
% given value using \let.

\def\pgfoolet#1#2{%
  \expandafter\global\expandafter\let\csname pgfooX\the\pgfoothis@count @#1\endcsname#2%
}%


% Add something to an attribute at the end
%
% #1 = attr
% #2 = code
%
% This method adds the give code to the attr at the end.

\def\pgfooappend#1#2{%
  \expandafter\expandafter\expandafter\def%
  \expandafter\expandafter\expandafter\pgf@oo@temp\expandafter\expandafter\expandafter{\csname pgfooX\the\pgfoothis@count @#1\endcsname#2}%
  \expandafter\global\expandafter\let\csname pgfooX\the\pgfoothis@count @#1\endcsname\pgf@oo@temp%
}%


% Add something to an attribute at the beginning
%
% #1 = attr
% #2 = code
%
% This method adds the give code to the attr at the beginning.

\def\pgfooprefix#1#2{%
  \pgfooget{#1}\pgf@oo@temp%
  \def\pgf@oo@@temp{#2}%
  \expandafter\expandafter\expandafter\def%
  \expandafter\expandafter\expandafter\pgf@oo@temp%
  \expandafter\expandafter\expandafter{\expandafter\pgf@oo@@temp\pgf@oo@temp}%
  \pgfoolet{#1}\pgf@oo@temp%
}%



% Help macro for scanning arguments
\newtoks\pgfoo@toks

\def\pgfoo@collect@args{%
  \pgfutil@ifnextchar\bgroup\pgfoo@collect@args@group\pgfoo@collect@args@nogroup%
}%
\def\pgfoo@collect@args@nogroup#1){%
  \pgfoo@toks{#1}%
  \pgfoo@continue%
}%
\def\pgfoo@collect@args@group#1{%
  \pgfutil@ifnextchar){\pgfoo@toks{{#1}}\expandafter\pgfoo@continue\pgfutil@gobble}{\pgfoo@collect@args@nogroup{#1}}%
}%


% Instantiate an object
%
% Possible syntax:
%
% 1) \pgfoonew new <class name>(<constructor parameters)
% 2) \pgfoonew \<objectname>=new <class name>(<constructor parameters)
% 3) \pgfoonew{attribute}=new <class name>(<constructor parameters)
%
% Description:
%
% Creates an object. After the object has been created, the method
% called <class name> (the constructor) is invoked. If the
% \<objectname>= part is present, the macro is assigned to the newly
% created object.

\def\pgfoonew{%
  \pgfutil@ifnextchar n{%
    \pgfoo@new\pgfoo@dummy=%
  }{%
    \pgfutil@ifnextchar\bgroup{%
      \pgfoo@new@attribute%
    }{%
      \pgfoo@new%
    }%
  }%
}%
\def\pgfoo@new#1={%
  \def\pgfutil@reserved@a{#1}%
  \let\pgfutil@next\pgfoo@@new
  \futurelet\pgfutil@let@token\pgfutil@ignorespaces}%
\def\pgfoo@@new new #1({%
  \expandafter\ifx\csname pgfooY#1.get id\endcsname\relax%
    \pgferror{Unknown class '#1'}%
  \else%
    \expandafter\pgfoo@new@create\pgfutil@reserved@a{#1}%
    {%
      \pgfoothis@count\pgfoo@objectcount%
      \let\pgfoo@attribute@op\pgfoolet
      \csname pgfooY#1.@pgfoo@process@attributes\endcsname%
    }%
    \let\pgfoo@continue=\pgf@oo@new@cont%
    \expandafter\pgfoo@collect@args%
  \fi%
}%

\def\pgf@oo@new@cont{%
  \expandafter\pgfoolastobj\expandafter.\expandafter i\expandafter n\expandafter i\expandafter t\expandafter(\the\pgfoo@toks)%
  \aftergroup\pgfoogc% cleanup after group
}%

\def\pgfoo@new@attribute#1={%
  \def\pgfutil@reserved@a{#1}%
  \let\pgfutil@next\pgfoo@@new@attribute
  \futurelet\pgfutil@let@token\pgfutil@ignorespaces}%
\def\pgfoo@@new@attribute new #1({%
  \expandafter\ifx\csname pgfooY#1.get id\endcsname\relax%
    \pgferror{Unknown class '#1'}%
  \else%
    \pgfoo@new@create\pgfoo@temp{#1}%
    {%
      \pgfoothis@count\pgfoo@objectcount%
      \let\pgfoo@attribute@op\pgfoolet%
      \csname pgfooY#1.@pgfoo@process@attributes\endcsname%
    }%
    \expandafter\pgfoolet\expandafter{\pgfutil@reserved@a}\pgfoo@temp%
    \let\pgfoo@continue=\pgf@oo@new@cont%
    \expandafter\pgfoo@collect@args%
  \fi%
}%

\def\pgfoo@new@create#1#2{%
  \advance\pgfoo@objectcount by 1\relax%
  \edef\pgfoolastobj{\noexpand\pgfoo@caller{\the\pgfoo@objectcount}}%
  \expandafter\gdef\csname pgfooX\the\pgfoo@objectcount @class\endcsname{#2}%
  \let#1\pgfoolastobj%
}%

\def\pgfoo@caller#1.#2({%
  \expandafter\let\expandafter\pgfoo@caller@temp\csname pgfooY\csname pgfooX#1@class\endcsname.#2\endcsname%
  \ifx\pgfoo@caller@temp\relax%
    % assume that #2 is of form "superclass.method"
    \expandafter\let\expandafter\pgfoo@caller@temp\csname pgfooY#2\endcsname%
    \ifx\pgfoo@caller@temp\relax%
      \pgferror{Object #1 has no method '#2'}%
    \fi%
  \fi%
  \def\pgf@marshal{%
    \pgfoothis@count=#1\relax%
  }%
  \let\pgfoo@continue\pgfoo@caller@cont%
  \pgfoo@collect@args%
}%
\def\pgfoo@caller@cont{%
  \edef\pgf@marshal{%
    \pgf@marshal%
    \noexpand\pgfoo@caller@temp(\the\pgfoo@toks)%
  }%
  \expandafter\pgf@marshal\expandafter\pgfoothis@count\the\pgfoothis@count\relax%
}%
\let\pgfoo@orig@caller=\pgfoo@caller

% The special "this" object

\def\pgfoothis.#1({%
  \expandafter\let\expandafter\pgfoo@caller@temp\csname pgfooY\csname pgfooX\the\pgfoothis@count @class\endcsname.#1\endcsname%
  \ifx\pgfoo@caller@temp\relax%
    % assume that #1 is of form "superclass.method"
    \expandafter\let\expandafter\pgfoo@caller@temp\csname pgfooY#1\endcsname%
  \fi%
  \let\pgfoo@continue\pgfoothis@cont%
  \pgfoo@collect@args%
}%
\def\pgfoothis@cont{%
  \expandafter\pgfoo@caller@temp\expandafter(\the\pgfoo@toks)%
}%



% Get the object id
%
% #1 = macro to store the id
%
% Description:
%
% This special method allows you to access the object id. You can then
% use \pgfoocall to call a method using this id. This is mainly useful
% when you wish to store the id for a longer time.

\def\pgfoo@id(#1){%
  \edef#1{\the\pgfoothis@count}%
}%

% Yields the object with the given id
%
% #1 = id
%
% Description:
%
% Given an object id, \pgfooobj{<id>} will yield the object having
% this id.

\def\pgfooobj#1{%
  \pgfoo@caller{#1}%
}%


% Get the object
%
% #1 = macro to store the object
%
% Description:
%
% This special method allows you to get a new handle to a given
% object. If \obj is an object, you could normally just say
% \let#1=\obj. However, if \obj happens to be \pgfoothis, then you may
% wish to get a handle to the object itself, not to the special macro
% \pgfoothis. In this case you can say \obj.get handle(#1).

\def\pgfoo@obj(#1){%
  \edef#1{\noexpand\pgfoo@caller{\the\pgfoothis@count}}%
}%



% The garbage collector
%
% Description:
%
% This method frees space occupied by unused objects. Garbage are
% objects that have been destroyed because of the end of the scope in
% which they were created. In this case, however, the memory used by
% this object is not immediately reused because the attributes of the
% object are actually stored in global variables. When the garbage
% collector is called, it will set all these global variables to
% \relax, thereby ensuring that no memory is needed for them.

\def\pgfoogc{%
  {%
    % We do this in a group...
    \pgfoothis@count\pgfoo@objectcount% this is temporary...
    \let\pgfoo@next=\pgfoo@dogc%
    \pgfoo@next%
  }%
}%
\def\pgfoo@dogc{%
  \advance\pgfoothis@count by 1\relax%
  \expandafter\ifx\csname pgfooX\the\pgfoothis@count @class\endcsname\relax%
    \let\pgfoo@next=\relax%
  \else%
    % Cleanup this object:
    % The following is the fast version of \pgfooobj{\the\pgfoo@objectcount}.@pgfoogc:
    \let\pgfoo@attribute@op\pgfoo@gc@attribute
    \csname pgfooY\csname pgfooX\the\pgfoothis@count @class\endcsname.@pgfoo@process@attributes\endcsname%
    \expandafter\global\expandafter\let\csname pgfooX\the\pgfoothis@count @class\endcsname\relax%
  \fi%
  \pgfoo@next%
}%



%
%
% Object class
% defines method "copy"
%
%

\pgfooclass{object}{%
  \method copy(#1) {%
    % create a new object
    \edef\pgfoo@temp@classname{\csname pgfooX\the\pgfoothis@count @class\endcsname}%
    \expandafter\pgfoo@new@create\expandafter#1\expandafter{\pgfoo@temp@classname}%
    \def\pgfoo@copy@attributes##1##2{%
      \pgfooget{##1}\pgfoo@attrvalue
      \expandafter\let\csname pgfooX\the\pgfoo@objectcount @##1\endcsname\pgfoo@attrvalue
    }%
    \let\pgfoo@attribute@op\pgfoo@copy@attributes
    \csname pgfooY\pgfoo@temp@classname .@pgfoo@process@attributes\endcsname
    \aftergroup\pgfoogc% cleanup after group
  }%
}%


%
%
% Signal class
%
%

\pgfooclass{signal}
{%
  %
  % This class implements signals.
  %
  % After you have created a signal object, you can call
  % connect to connect a slot. Then, whenever the emit method is
  % called, all connected methods get called.

  % Attribute
  \attribute emitter;%
  % Collects the objects that should be called.

  % Constructor
  \method signal() {}%

  % Connect a slot (method) #2 of object #1
  \method connect(#1,#2) {%
    {%
      #1.get id(\pgf@tempid)%
      % Save in emitter
      \pgfooget{emitter}\pgf@temp%
      \let\pgfoo@signal@call=\relax% avoid expansion
      \edef\pgf@temp{\pgf@temp\pgfoo@signal@call{\pgf@tempid}{#2}}%
      \pgfoolet{emitter}\pgf@temp%
    }%
  }%

  \def\pgfoo@signal@call#1#2{%
    \def\pgf@temp{\pgfooobj{#1}.#2}%
    \expandafter\pgf@temp\expandafter(\pgfoo@signal@args)%
  }%

  % Emit a signal
  \method emit(#1) {%
    \def\pgfoo@signal@args{#1}%
    \pgfoovalueof{emitter}
  }%
}%




\endinput
